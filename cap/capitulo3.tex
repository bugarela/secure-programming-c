\chapter{Comparativo}
\label{cap3}

Para cada ferramenta a ser comparada, executou-se um processo de testes para detecção de falhas de segurança nos códigos não conformes listados no Capítulo \ref{cap2}. A seguir, serão expostos os resultados obtidos, assim como comparações relevantes.

\section{Método de aplicação da ferramenta}

\subsection{Cppcheck}

A execução dos testes com a ferramenta Cppcheck demanda seu download e instalação, feito a partir da fonte oficial \cite{cppcheck}. Os códigos não conformes foram organizados em um diretório local, e a ferramenta permitiu a verificação a partir do caminho raiz do diretório criado. Assim, uma única execução foi suficiente para obter os resultados com erros apontados para todos os arquivos de código fonte. Cada alerta informa o arquivo e a linha do código potencialmente inseguro.

\subsection{FlawFinder}

O funcionamento da ferramenta FlawFinder proporciona um procedimento semelhante ao feito para o analisador Cppcheck. O formato da saída é equivalente, e ela, da mesma forma, permite a verificação de um diretório. Assim, o mesmo diretório de códigos fonte não conformes foi utilizado. Os preparativos envolvendo download e instalação foram feitas de forma semelhante, a partir da fonte oficial \cite{flawfinder}.

\subsection{LGTM}

A ferramenta LGTM apresentou demandas disparadamente diferentes. Este analisador funciona como um serviço, se conectando como uma aplicação de outros serviços como o GitHub. Assim, se tornou necessário que os códigos analisados estivessem disponíveis em tal serviço.

O serviço LGTM não permite a verificação de um diretório, e sim de um projeto. Isso se torna relevante por duas diferenças de estrutura: A necessidade de um arquivo para construção (\textit{Makefile}) e de remoção de conflitos.

Os exemplos de não conformidade providos pela CERT-C são colocados de maneira isolada. Assim, muitos deles possuem nomes idênticos para funções, e todos implementam uma função \texttt{main}. Esses nomes foram alterados para nomes arbitrários não conflitantes, afim de permitir a compilação do projeto criado. A especificação no arquivo de construção determina que todos os arquivos de código fonte devem ser compilados.

Com o projeto sem conflitos e com arquivo de construção disponível em um repositório do GitHub configurado com a aplicação LGTM, é possível abrir um \textit{Pull Request} e receber uma verificação automática, com os arquivos e linhas correspondentes ao alerta listados na página \textit{web} do serviço.


\section{Resultados}
Com as informações resultantes da execução da ferramenta, executou-se um processo de análise para identificar se a falha apontada é relacionada com a determinação da regra imposta pela CERT-C. A Tabela \ref{tab:all} indica quais ferramentas apontaram falhas referentes a quais regras de prioridade 1 - representado com S nas células onde a ferramenta da coluna foi capaz de detectar não conformidade para a regra da linha, e N caso a ferramenta não tenha apontado a falha em questão.

\begin{table}[]
\begin{tabular}{@{}cccc@{}}
\toprule
\textbf{Regra} & \multicolumn{3}{c}{\textbf{Detecção (S/N)}} \\
               & Cppcheck & FlawFinder & LGTM \\ \midrule
EXP33-C        & N        & N          & N    \\ \midrule
EXP34-C        & N        & N          & N    \\ \midrule
ARR38-C        & N        & S          & N    \\ \midrule
STR31-C        & S        & S          & N    \\ \midrule
STR32-C        & N        & S          & N    \\ \midrule
STR38-C        & N        & S          & N    \\ \midrule
MEM30-C        & N        & N          & N    \\ \midrule
MEM34-C        & N        & N          & N    \\ \midrule
FIO30-C        & N        & S          & N    \\ \midrule
FIO34-C        & N        & N          & N    \\ \midrule
FIO37-C        & N        & S          & N    \\ \midrule
ENV32-C        & N        & N          & N    \\ \midrule
ENV33-C        & N        & S          & N    \\ \midrule
SIG30-C        & N        & N          & N    \\ \midrule
ERR33-C        & N        & N          & N    \\ \midrule
MSC32-C        & N        & N          & N    \\ \midrule
MSC33-C        & S        & N          & S    \\ \bottomrule
\end{tabular}
\caption{Detecção de conformidade com cada regra por cada ferramenta}
\label{tab:all}
\end{table}

Ainda, os resultados foram analisados em busca de falsos positivos. Nesse contexto, um falso positivo é tido como uma falha apontada que não tem uma relação direta com o problema. Em algumas ocasiões, as ferramentas apontam falhas que são consequências indiretas da não conformidade - o que foi considerado um falso positivo com a justificativa de que um programador fazendo correções referentes aos alertas tenderia a resolver o problema errado. Caso o alerta aponte uma situação que, se corrigida, torna o código conforme, ele é considerado uma detecção correta.

Com essa análise, é possível construir matrizes de confusão para as ferramentas. Uma matriz de confusão simples possui quatro quadrantes:
\begin{itemize}
  \item Verdadeiro positivo: Falhas detectadas correspondem à não conformidade.
  \item Falso negativo: A não conformidade não foi apontada como falha.
  \item Falso positivo: Falhas detectadas não correspondem à não conformidade.
  \item Verdadeiro negativo: Não foram detectadas falhas não correspondentes.
\end{itemize}

Destaca-se a diferenças entre as linhas da matriz de confusão para esse contexto: todos os códigos analisados possuem não conformidade com alguma regra, e falsos positivos e verdadeiros negativos correspondem a detecção ou não de falhas não correspondentes com a não conformidade.

\begin{table}[ht]
  \begin{tabular}{@{}lcc@{}}
  \toprule
                      & \textbf{Detecção de falha} & \multicolumn{1}{l}{\textbf{Não deteccção de falha}} \\ \midrule
  \textbf{Falha correspondente} & 2                        & 15                                               \\
  \textbf{Restante do código} & 0                        & 17                                               \\ \bottomrule
  \end{tabular}
  \caption{Matriz de confusão para detecção com Cppcheck}
  \label{tab:cppcheck}
  \end{table}

\begin{table}[ht]
\begin{tabular}{@{}lcc@{}}
\toprule
                    & \textbf{Detecção de falha} & \multicolumn{1}{l}{\textbf{Não deteccção de falha}} \\ \midrule
\textbf{Falha correspondente} & 7                        & 10                                               \\
\textbf{Restante do código} & 4                        & 13                                               \\ \bottomrule
\end{tabular}
\caption{Matriz de confusão para detecção com FlawFinder}
\label{tab:flawfinder}
\end{table}

\begin{table}[ht]
  \begin{tabular}{@{}lcc@{}}
  \toprule
                      & \textbf{Detecção de falha} & \multicolumn{1}{l}{\textbf{Não deteccção de falha}} \\ \midrule
  \textbf{Falha correspondente} & 1                        & 14                                               \\
  \textbf{Restante do código} & 0                        & 17                                               \\ \bottomrule
  \end{tabular}
  \caption{Matriz de confusão para detecção com LGTM}
  \label{tab:lgtm}
  \end{table}

As matrizes de confusão são apresentadas por ferramenta, sendo a matriz para os resultados de Cppcheck dispostos na Tabela \ref{tab:cppcheck}, de FlawFinder na Tabela \ref{tab:flawfinder} e de LGTM na Tabela \ref{tab:lgtm}.

A partir da observação desses resultados com base no critério proposto pelo padrão \cite{ccert} de quantidade de não conformidades detectadas, a ferramenta FlawFinder se mostra mais adepta para análise de código C/C++ com objetivo de mitigar vulnerabilidades abordadas nas regras de nível 1 do padrão CERT C.
