\documentclass{beamer}

% \usepackage{graphicx,hyperref,udesc,url}
\usepackage{graphicx,udesc,url}
\usepackage[utf8]{inputenc}
\usepackage[T1]{fontenc}
\usepackage{booktabs}
\usepackage{gensymb}
\usepackage{amsmath,amssymb,proof}
\usepackage{booktabs}
\usepackage{dirtree}
\usepackage{listings}

\definecolor{green-hl}{RGB}{54,88,65}
\newcommand{\hl}[1]{{\color{white}\colorbox{green-hl}{#1}}}

\lstset{
    numbers=left,                   % where to put the line-numbers
    numberstyle=\small \ttfamily \color[rgb]{0.4,0.4,0.4},
                % style used for the linenumbers
    showspaces=false,               % show spaces adding special underscores
    showstringspaces=false,         % underline spaces within strings
    showtabs=false,                 % show tabs within strings adding particular underscores
    frame=lines,                    % add a frame around the code
    tabsize=4,                        % default tabsize: 4 spaces
    breaklines=false,                % automatic line breaking
    breakatwhitespace=false,        % automatic breaks should only happen at whitespace
    basicstyle=\ttfamily,
    %identifierstyle=\color[rgb]{0.3,0.133,0.133},   % colors in variables and function names, if desired.
    keywordstyle=\color[rgb]{0.133,0.133,0.6},
    commentstyle=\color[rgb]{0.133,0.545,0.133},
    stringstyle=\color[rgb]{0.627,0.126,0.941},
}

\sloppy

\title[Analisadores de código para o padrão CERT C]{Estudo comparativo entre 3 analisadores de código C sobre as
ameaças de nível 1 do padrão CERT C 2016}

\author[André, Gabriela, Lucas]{
    André Eduardo Pacheco Dias\smallskip \leavevmode \\
    {\scriptsize Universidade do Estado de Santa Catarina \\\smallskip
    \url{andre.dias@msn.com}  \\}
    \medskip
    Gabriela Moreira Mafra\\\smallskip
    {\scriptsize Universidade do Estado de Santa Catarina \\\smallskip
    \url{gabrielamoreiramafra@gmail.com}  \\ }
    \medskip
    Lucas Schmitt Seidel\\\smallskip
    {\scriptsize Universidade do Estado de Santa Catarina \\\smallskip\vspace{-2mm}
    \url{lucasschmittseidel@gmail.com}}
    \vspace{-7mm}
}

\begin{document}
  \date{24 de Junho de 2019}
  \begin{frame}
      \titlepage
  \end{frame}

\tableofcontents

\section{Objetivo}

\begin{frame}
  \frametitle{Programação segura}

  \begin{itemize}
    \item Prática
    \item Mitigar vulnerabilidades
    \begin{itemize}
      \item Geradas por erros de programação
    \end{itemize}
    \item Importância crescente devido ao aumento de ativos resguardados por \textit{software}.
  \end{itemize}\pause

  \begin{block}{CERT C}
    Padrão de programação segura em C/C++
    \begin{itemize}
      \item Dividido em Regras
      \item Classifica por prioridade
      \begin{itemize}
        \item Grau de severidade
        \item Probabilidade de exploração
        \item Custo de remediação
      \end{itemize}
    \end{itemize}
 \end{block}

\end{frame}

\begin{frame}
  \frametitle{CERT C}
  \begin{table}[!h]
    \hspace*{-16px}
    \begin{tabular}{@{}lll@{}}
      Nível & Prioridades & Possível Interpretação \\\toprule
      L1 & 12, 18, 27 & Alta severidade, muito provável, sem custo de reparo\\
      L2 & 6, 8, 9 & Média severidade, provável, custo médio de reparo\\
      L3 & 1, 2, 3, 4 & Baixa severidade, pouco provável, custo alto de reparo\\
    \end{tabular}
    \caption{Níveis de prioridades}
    \label{table:prioriades}
  \end{table}\pause

  \begin{block}{Escopo}
    Neste trabalho, foram analisadas apenas as regras de nível 1 (L1). Isso inclui 17 regras.
 \end{block}

\end{frame}

\begin{frame}
  \frametitle{Analisadores de código}

  Para evitar falha humana e economizar recurso, é interessante usar um analisador automático para encontrar falhas.\\\medskip

  O padrão CERT C indica que seja usada a ferramenta capaz de detectar o maior número de não conformidades possível.

\end{frame}

\begin{frame}
  \frametitle{Objetivo}

  Este trabalho tem como objetivo \hl{comparar} três ferramentas analisadoras com cógido aberto recomendadas pela OWASP: Cppcheck, FlawFinder e LGTM; de acordo com sua \hl{detecção de não conformidades} com as regras de nível 1 do padrão CERT C 2016.

\end{frame}

\section{Analisadores de código}

\begin{frame}
  \frametitle{Cppcheck}

  \begin{minipage}{0.5\textwidth}\centering
    \begin{itemize}
      \item Objetivo de minimizar falsos positivos.
      \item Disponível para download, é executada sobre um diretório local.
      \item Aponta falhas de segurança e problemas de estilo de código.
    \end{itemize}
  \end{minipage}
  \hfill
  \begin{minipage}{0.4\textwidth}\raggedleft

  \includegraphics[scale=0.33]{img/cppcheck.png}

  \end{minipage}

\end{frame}

\begin{frame}
  \frametitle{FlawFinder}

  \begin{minipage}{0.5\textwidth}\centering
    \begin{itemize}
      \item Reporta possíveis fragilidades de segurança ordenadas por um nível de vulnerabilidade.
      \item Disponível para download, é executada sobre um diretório local.
      \item Reconhecida pela CWE e
      \textit{CII Best Practices}.
    \end{itemize}
  \end{minipage}
  \hfill
  \begin{minipage}{0.4\textwidth}\raggedleft

  \includegraphics[scale=0.6]{img/flawfinder.png}

  \end{minipage}

\end{frame}

\begin{frame}
  \frametitle{LGTM}

  \begin{minipage}{0.5\textwidth}\centering
    \begin{itemize}
      \item Analisa a semântica do código com métodos da ciência de dados.
      \item Procura por variantes no histórico do projeto.
      \item Serviço que se conecta como uma aplicação do GitHub.
      \item Suporta várias linguagens.
    \end{itemize}
  \end{minipage}
  \hfill
  \begin{minipage}{0.4\textwidth}\raggedleft

  \includegraphics[scale=0.6]{img/lgtm.png}

  \end{minipage}

\end{frame}

\section{Regras}

\begin{frame}
  \frametitle{Regras}

  As regras impostas pelo padrão CERT C são definidas com os elementos:
  \begin{itemize}
    \item Descrição da prática e explicação das consequências
    \item Exemplo(s) de código(s) não conforme(s)
    \item Versão(ões) conforme(s) do(s) mesmo(s)
  \end{itemize}

\end{frame}

\begin{frame}
  \frametitle{Método de aplicação}

  \begin{minipage}{0.5\textwidth}\centering
    Para cada regra, escolheu-se arbitrariamente um exemplo não conforme para ser analisado.\\\bigskip

  O conjunto de exemplos não conformes foi agrupado em um diretório ou projeto, e enviado à ferramenta de análise.
  \end{minipage}
  \hfill
  \begin{minipage}{0.4\textwidth}\raggedright

    \dirtree{%
    .1 projeto.
    .2 ARR38-C.c.
    .2 ENV32-C.c.
    .2 ENV33-C.c.
    .2 ERR33-C.c.
    .2 EXP33-C.c.
    .2 EXP34-C.c.
    .2 FIO30-C.c.
    .2 FIO34-C.c.
    .2 FIO37-C.c.
    .2 MEM30-C.c.
    .2 MEM34-C.c.
    .2 MSC32-C.c.
    .2 MSC33-C.c.
    .2 SIG30-C.c.
    .2 STR31-C.c.
    .2 STR32-C.c.
    .2 STR38-C.c.
    }
  \end{minipage}
\end{frame}

\begin{frame}
  \frametitle{Exemplo: Regra STR31-C (1/2)}
  \begin{figure}[h!]
    \centering
    \lstinputlisting[language=C]{"code/STR31-C.c"}
    \caption{Código não conforme para a regra STR31-C}
  \label{fig:STR31-C}
  \end{figure}
\end{frame}

\begin{frame}
  \frametitle{Exemplo: Regra STR31-C (2/2)}
  \textbf{\Large Alerta FlawFinder:}\\\medskip
  STR31-C.c:2:  [2] (buffer) char:
  Statically-sized arrays can be improperly restricted, leading to potential overflows or other issues (CWE-119!/CWE-120). Perform bounds checking, use functions that limit length, or ensure that the size is larger than the maximum possible length.\\\bigskip

  \textbf{\Large Alerta Cppcheck:}\\\medskip
  [STR31-C.c:5]: (style) Array index 'i' is used before limits check.

\end{frame}


\begin{frame}
  \frametitle{Exemplo: Regra FIO30-C (1/2)}
  \vspace{-4mm}
  \begin{figure}[h!]
    \centering
    \lstinputlisting[language=C]{"code/FIO30-C.c"}
    \caption{Código não conforme para a regra FIO30-C}
  \label{fig:FIO30-C}
  \end{figure}
\end{frame}

\begin{frame}
  \frametitle{Exemplo: Regra FIO30-C (2/2)}
  \textbf{\Large Alerta FlawFinder:}\\\medskip
  FIO30-C.c:24: [4] (format) fprintf:
  If format strings can be influenced by an attacker, they can be exploited (CWE-134). Use a constant for the format specification.
\end{frame}

\section{Resultados e Considerações}

\begin{frame}
  \frametitle{Não conformidades detectadas por ferramenta (1/2)}
  \begin{table}[]
    \begin{tabular}{@{}cccc@{}}
    \toprule
    \textbf{Regra} & \multicolumn{3}{c}{\textbf{Detecção (S/N)}} \\
                   & Cppcheck & FlawFinder & LGTM \\ \midrule
    EXP33-C        & N        & N          & N    \\ \midrule
    EXP34-C        & N        & N          & N    \\ \midrule
    ARR38-C        & N        & S          & N    \\ \midrule
    STR31-C        & S        & S          & N    \\ \midrule
    STR32-C        & N        & S          & N    \\ \midrule
    STR38-C        & N        & S          & N    \\ \midrule
    MEM30-C        & N        & N          & N    \\ \midrule
    MEM34-C        & N        & N          & N    \\ \midrule
    FIO30-C        & N        & S          & N    \\ \bottomrule
    \end{tabular}
    \end{table}

\end{frame}

\begin{frame}
  \frametitle{Não conformidades detectadas por ferramenta (2/2)}
  \begin{table}[]
    \begin{tabular}{@{}cccc@{}}
    \toprule
    \textbf{Regra} & \multicolumn{3}{c}{\textbf{Detecção (S/N)}} \\
                   & Cppcheck & FlawFinder & LGTM \\ \midrule
    FIO34-C        & N        & N          & N    \\ \midrule
    FIO37-C        & N        & S          & N    \\ \midrule
    ENV32-C        & N        & N          & N    \\ \midrule
    ENV33-C        & N        & S          & N    \\ \midrule
    SIG30-C        & N        & N          & N    \\ \midrule
    ERR33-C        & N        & N          & N    \\ \midrule
    MSC32-C        & N        & N          & N    \\ \midrule
    MSC33-C        & S        & N          & S    \\ \bottomrule
    \end{tabular}
    \end{table}
\end{frame}

\begin{frame}
  \frametitle{Matriz de confusão: Cppcheck}
  \begin{table}[ht]
    \hspace*{-19px}
    \begin{tabular}{@{}lcc@{}}
    \toprule
                        & \textbf{Detecção de falha} & \multicolumn{1}{l}{\textbf{Não deteccção de falha}} \\ \midrule
    \textbf{Falha correspondente} & 2                        & 15                                               \\
    \textbf{Restante do código} & 0                        & 17                                               \\ \bottomrule
    \end{tabular}
    \caption{Matriz de confusão para detecção com Cppcheck}
    \label{tab:cppcheck}
    \end{table}
\end{frame}

\begin{frame}
  \frametitle{Matriz de confusão: FlawFinder}
  \begin{table}[ht]
    \hspace*{-19px}
    \begin{tabular}{@{}lcc@{}}
    \toprule
                        & \textbf{Detecção de falha} & \multicolumn{1}{l}{\textbf{Não deteccção de falha}} \\ \midrule
    \textbf{Falha correspondente} & 7                        & 10                                               \\
    \textbf{Restante do código} & 4                        & 13                                               \\ \bottomrule
    \end{tabular}
    \caption{Matriz de confusão para detecção com FlawFinder}
    \label{tab:flawfinder}
    \end{table}
\end{frame}

\begin{frame}
  \frametitle{Matriz de confusão: LGTM}
  \begin{table}[ht]
    \hspace*{-19px}
    \begin{tabular}{@{}lcc@{}}
    \toprule
                        & \textbf{Detecção de falha} & \multicolumn{1}{l}{\textbf{Não deteccção de falha}} \\ \midrule
    \textbf{Falha correspondente} & 1                        & 14                                               \\
    \textbf{Restante do código} & 0                        & 17                                               \\ \bottomrule
    \end{tabular}
    \caption{Matriz de confusão para detecção com LGTM}
    \label{tab:lgtm}
    \end{table}
\end{frame}

\begin{frame}
  \frametitle{Considerações}
  Verdadeiros positivos e falsos negativos são mais relevantes devido ao critério do padrão CERT C.\\\medskip

  A ferramenta FlawFinder se mostrou mais adepta para análise de código C/C++ com objetivo de mitigar vulnerabilidades abordadas nas regras de nível 1 do padrão CERT C 2016.\pause

  \begin{block}{Limitações}
    \begin{itemize}
      \item Foram considerados apenas os códigos não conformes.
      \item As análises envolveram apenas um dos exemplos de não conformidade.
    \end{itemize}

  \end{block}
\end{frame}

\begin{frame}
  \frametitle{}
  {\Huge Fim :D}\\\bigskip

  André Eduardo Pacheco Dias\\
  Gabriela Moreira Mafra\\
  Lucas Schmitt Seidel

\end{frame}

\end{document}
